% Options for packages loaded elsewhere
\PassOptionsToPackage{unicode}{hyperref}
\PassOptionsToPackage{hyphens}{url}
%
\documentclass[
]{article}
\usepackage{amsmath,amssymb}
\usepackage{lmodern}
\usepackage{iftex}
\ifPDFTeX
  \usepackage[T1]{fontenc}
  \usepackage[utf8]{inputenc}
  \usepackage{textcomp} % provide euro and other symbols
\else % if luatex or xetex
  \usepackage{unicode-math}
  \defaultfontfeatures{Scale=MatchLowercase}
  \defaultfontfeatures[\rmfamily]{Ligatures=TeX,Scale=1}
\fi
% Use upquote if available, for straight quotes in verbatim environments
\IfFileExists{upquote.sty}{\usepackage{upquote}}{}
\IfFileExists{microtype.sty}{% use microtype if available
  \usepackage[]{microtype}
  \UseMicrotypeSet[protrusion]{basicmath} % disable protrusion for tt fonts
}{}
\makeatletter
\@ifundefined{KOMAClassName}{% if non-KOMA class
  \IfFileExists{parskip.sty}{%
    \usepackage{parskip}
  }{% else
    \setlength{\parindent}{0pt}
    \setlength{\parskip}{6pt plus 2pt minus 1pt}}
}{% if KOMA class
  \KOMAoptions{parskip=half}}
\makeatother
\usepackage{xcolor}
\usepackage[margin=1in]{geometry}
\usepackage{color}
\usepackage{fancyvrb}
\newcommand{\VerbBar}{|}
\newcommand{\VERB}{\Verb[commandchars=\\\{\}]}
\DefineVerbatimEnvironment{Highlighting}{Verbatim}{commandchars=\\\{\}}
% Add ',fontsize=\small' for more characters per line
\usepackage{framed}
\definecolor{shadecolor}{RGB}{248,248,248}
\newenvironment{Shaded}{\begin{snugshade}}{\end{snugshade}}
\newcommand{\AlertTok}[1]{\textcolor[rgb]{0.94,0.16,0.16}{#1}}
\newcommand{\AnnotationTok}[1]{\textcolor[rgb]{0.56,0.35,0.01}{\textbf{\textit{#1}}}}
\newcommand{\AttributeTok}[1]{\textcolor[rgb]{0.77,0.63,0.00}{#1}}
\newcommand{\BaseNTok}[1]{\textcolor[rgb]{0.00,0.00,0.81}{#1}}
\newcommand{\BuiltInTok}[1]{#1}
\newcommand{\CharTok}[1]{\textcolor[rgb]{0.31,0.60,0.02}{#1}}
\newcommand{\CommentTok}[1]{\textcolor[rgb]{0.56,0.35,0.01}{\textit{#1}}}
\newcommand{\CommentVarTok}[1]{\textcolor[rgb]{0.56,0.35,0.01}{\textbf{\textit{#1}}}}
\newcommand{\ConstantTok}[1]{\textcolor[rgb]{0.00,0.00,0.00}{#1}}
\newcommand{\ControlFlowTok}[1]{\textcolor[rgb]{0.13,0.29,0.53}{\textbf{#1}}}
\newcommand{\DataTypeTok}[1]{\textcolor[rgb]{0.13,0.29,0.53}{#1}}
\newcommand{\DecValTok}[1]{\textcolor[rgb]{0.00,0.00,0.81}{#1}}
\newcommand{\DocumentationTok}[1]{\textcolor[rgb]{0.56,0.35,0.01}{\textbf{\textit{#1}}}}
\newcommand{\ErrorTok}[1]{\textcolor[rgb]{0.64,0.00,0.00}{\textbf{#1}}}
\newcommand{\ExtensionTok}[1]{#1}
\newcommand{\FloatTok}[1]{\textcolor[rgb]{0.00,0.00,0.81}{#1}}
\newcommand{\FunctionTok}[1]{\textcolor[rgb]{0.00,0.00,0.00}{#1}}
\newcommand{\ImportTok}[1]{#1}
\newcommand{\InformationTok}[1]{\textcolor[rgb]{0.56,0.35,0.01}{\textbf{\textit{#1}}}}
\newcommand{\KeywordTok}[1]{\textcolor[rgb]{0.13,0.29,0.53}{\textbf{#1}}}
\newcommand{\NormalTok}[1]{#1}
\newcommand{\OperatorTok}[1]{\textcolor[rgb]{0.81,0.36,0.00}{\textbf{#1}}}
\newcommand{\OtherTok}[1]{\textcolor[rgb]{0.56,0.35,0.01}{#1}}
\newcommand{\PreprocessorTok}[1]{\textcolor[rgb]{0.56,0.35,0.01}{\textit{#1}}}
\newcommand{\RegionMarkerTok}[1]{#1}
\newcommand{\SpecialCharTok}[1]{\textcolor[rgb]{0.00,0.00,0.00}{#1}}
\newcommand{\SpecialStringTok}[1]{\textcolor[rgb]{0.31,0.60,0.02}{#1}}
\newcommand{\StringTok}[1]{\textcolor[rgb]{0.31,0.60,0.02}{#1}}
\newcommand{\VariableTok}[1]{\textcolor[rgb]{0.00,0.00,0.00}{#1}}
\newcommand{\VerbatimStringTok}[1]{\textcolor[rgb]{0.31,0.60,0.02}{#1}}
\newcommand{\WarningTok}[1]{\textcolor[rgb]{0.56,0.35,0.01}{\textbf{\textit{#1}}}}
\usepackage{graphicx}
\makeatletter
\def\maxwidth{\ifdim\Gin@nat@width>\linewidth\linewidth\else\Gin@nat@width\fi}
\def\maxheight{\ifdim\Gin@nat@height>\textheight\textheight\else\Gin@nat@height\fi}
\makeatother
% Scale images if necessary, so that they will not overflow the page
% margins by default, and it is still possible to overwrite the defaults
% using explicit options in \includegraphics[width, height, ...]{}
\setkeys{Gin}{width=\maxwidth,height=\maxheight,keepaspectratio}
% Set default figure placement to htbp
\makeatletter
\def\fps@figure{htbp}
\makeatother
\setlength{\emergencystretch}{3em} % prevent overfull lines
\providecommand{\tightlist}{%
  \setlength{\itemsep}{0pt}\setlength{\parskip}{0pt}}
\setcounter{secnumdepth}{-\maxdimen} % remove section numbering
\ifLuaTeX
  \usepackage{selnolig}  % disable illegal ligatures
\fi
\IfFileExists{bookmark.sty}{\usepackage{bookmark}}{\usepackage{hyperref}}
\IfFileExists{xurl.sty}{\usepackage{xurl}}{} % add URL line breaks if available
\urlstyle{same} % disable monospaced font for URLs
\hypersetup{
  pdftitle={Entrega1ercorte},
  pdfauthor={Erik Castillo-Nicolas Cordoba - Daniel Caicedo},
  hidelinks,
  pdfcreator={LaTeX via pandoc}}

\title{Entrega1ercorte}
\author{Erik Castillo-Nicolas Cordoba - Daniel Caicedo}
\date{2022-09-05}

\begin{document}
\maketitle

En este informe desarrollaremos los classwork correspondientes al primer
corte de la materia electiva del area electronica.

Inicialmente procedemos a llamar las librerias correspondientes donde se
encuentran los dataframes para consultar la informacion propuesta en el
ejercicio.

Numeros del 1 al 100:

\begin{Shaded}
\begin{Highlighting}[]
\DecValTok{1}\SpecialCharTok{:}\DecValTok{100}
\end{Highlighting}
\end{Shaded}

\begin{verbatim}
##   [1]   1   2   3   4   5   6   7   8   9  10  11  12  13  14  15  16  17  18
##  [19]  19  20  21  22  23  24  25  26  27  28  29  30  31  32  33  34  35  36
##  [37]  37  38  39  40  41  42  43  44  45  46  47  48  49  50  51  52  53  54
##  [55]  55  56  57  58  59  60  61  62  63  64  65  66  67  68  69  70  71  72
##  [73]  73  74  75  76  77  78  79  80  81  82  83  84  85  86  87  88  89  90
##  [91]  91  92  93  94  95  96  97  98  99 100
\end{verbatim}

para imprimir los numeros del 1 al 100 debemos ubicar solo este comando
y ejecutarlo y si queremos realizarlo en matriz podemos ponerlo asi:

\begin{Shaded}
\begin{Highlighting}[]
\NormalTok{x}\OtherTok{\textless{}{-}}\FunctionTok{c}\NormalTok{(}\DecValTok{1}\SpecialCharTok{:}\DecValTok{100}\NormalTok{)}
\FunctionTok{print}\NormalTok{ (x)}
\end{Highlighting}
\end{Shaded}

\begin{verbatim}
##   [1]   1   2   3   4   5   6   7   8   9  10  11  12  13  14  15  16  17  18
##  [19]  19  20  21  22  23  24  25  26  27  28  29  30  31  32  33  34  35  36
##  [37]  37  38  39  40  41  42  43  44  45  46  47  48  49  50  51  52  53  54
##  [55]  55  56  57  58  59  60  61  62  63  64  65  66  67  68  69  70  71  72
##  [73]  73  74  75  76  77  78  79  80  81  82  83  84  85  86  87  88  89  90
##  [91]  91  92  93  94  95  96  97  98  99 100
\end{verbatim}

En este punto inician los classwork: Para iniciar con el trabajo sobre
los dataframes del aeropuerto, vuelos, y demas debemos llamar las
librerias correspondientes:

\begin{Shaded}
\begin{Highlighting}[]
\FunctionTok{library}\NormalTok{(nycflights13)}
\FunctionTok{library}\NormalTok{(tidyverse)}
\end{Highlighting}
\end{Shaded}

\begin{verbatim}
## -- Attaching packages --------------------------------------- tidyverse 1.3.2 --
## v ggplot2 3.3.6     v purrr   0.3.4
## v tibble  3.1.8     v dplyr   1.0.9
## v tidyr   1.2.0     v stringr 1.4.0
## v readr   2.1.2     v forcats 0.5.2
## -- Conflicts ------------------------------------------ tidyverse_conflicts() --
## x dplyr::filter() masks stats::filter()
## x dplyr::lag()    masks stats::lag()
\end{verbatim}

\begin{Shaded}
\begin{Highlighting}[]
\NormalTok{nycflights13\_al }\OtherTok{\textless{}{-}}\NormalTok{ nycflights13}\SpecialCharTok{::}\NormalTok{airlines }\CommentTok{\#aerolineas}
\NormalTok{nycflights13\_ap }\OtherTok{\textless{}{-}}\NormalTok{ nycflights13}\SpecialCharTok{::}\NormalTok{airports }\CommentTok{\#aeropuertos}
\NormalTok{nycflights13\_vu }\OtherTok{\textless{}{-}}\NormalTok{ nycflights13}\SpecialCharTok{::}\NormalTok{flights }\CommentTok{\#vuelos}
\NormalTok{nycflights13\_av }\OtherTok{\textless{}{-}}\NormalTok{ nycflights13}\SpecialCharTok{::}\NormalTok{planes }\CommentTok{\#aviones}
\NormalTok{nycflights13\_cl }\OtherTok{\textless{}{-}}\NormalTok{ nycflights13}\SpecialCharTok{::}\NormalTok{weather }\CommentTok{\#clima}

\FunctionTok{colnames}\NormalTok{(nycflights13\_vu)}
\end{Highlighting}
\end{Shaded}

\begin{verbatim}
##  [1] "year"           "month"          "day"            "dep_time"      
##  [5] "sched_dep_time" "dep_delay"      "arr_time"       "sched_arr_time"
##  [9] "arr_delay"      "carrier"        "flight"         "tailnum"       
## [13] "origin"         "dest"           "air_time"       "distance"      
## [17] "hour"           "minute"         "time_hour"
\end{verbatim}

Se asignan toods los dataframes que componen la libreria, aunque
particularmente trabajaremos dos o tres de ellos, aun asi contaremos con
toda la informacion en nuestros dataframes locales al realizar esta
operacion

\hypertarget{uso-de-los-filters}{%
\subsection{USO DE LOS FILTERS}\label{uso-de-los-filters}}

Para el primer ejercicio nos solicita encontrar los aviones que:

Tuvo un retraso de llegada de dos o más horas.

\begin{Shaded}
\begin{Highlighting}[]
\CommentTok{\#Ejercicio 5.2.4}
\StringTok{\textquotesingle{}5.2.4{-}1.1\textquotesingle{}}\OtherTok{\textless{}{-}}\FunctionTok{filter}\NormalTok{ (nycflights13\_vu, arr\_delay }\SpecialCharTok{\textgreater{}} \StringTok{\textquotesingle{}2\textquotesingle{}}\NormalTok{ )}
\end{Highlighting}
\end{Shaded}

Voló a Houston ( IAHo HOU)

\begin{Shaded}
\begin{Highlighting}[]
\StringTok{\textquotesingle{}5.2.4{-}1.2\textquotesingle{}}\OtherTok{\textless{}{-}}\FunctionTok{filter}\NormalTok{ (nycflights13\_vu, dest }\SpecialCharTok{==} \StringTok{"IAH"}\SpecialCharTok{|}\NormalTok{ dest }\SpecialCharTok{==}\StringTok{"HOU"}\NormalTok{)}
\end{Highlighting}
\end{Shaded}

Fueron operados por United, American o Delta

\begin{Shaded}
\begin{Highlighting}[]
\StringTok{\textquotesingle{}5.2.4{-}1.3\textquotesingle{}}\OtherTok{\textless{}{-}}\FunctionTok{filter}\NormalTok{ (nycflights13\_vu, carrier }\SpecialCharTok{==} \StringTok{"AA"}\SpecialCharTok{|}\NormalTok{ dest }\SpecialCharTok{==}\StringTok{"UA"} \SpecialCharTok{|}\NormalTok{ dest }\SpecialCharTok{==}\StringTok{"DL"}\NormalTok{) }\CommentTok{\#primero se consulta la tabla de aerolineas para sacar las siglas de aerolinea}
\end{Highlighting}
\end{Shaded}

Salida en verano (julio, agosto y septiembre)

\begin{Shaded}
\begin{Highlighting}[]
\StringTok{\textquotesingle{}5.2.4{-}1.4\textquotesingle{}}\OtherTok{\textless{}{-}}\FunctionTok{filter}\NormalTok{ (nycflights13\_vu, month}\SpecialCharTok{==}\DecValTok{7}\SpecialCharTok{|}\NormalTok{ month}\SpecialCharTok{==}\DecValTok{8} \SpecialCharTok{|}\NormalTok{ month}\SpecialCharTok{==}\DecValTok{9}\NormalTok{) }
\end{Highlighting}
\end{Shaded}

Llegó más de dos horas tarde, pero no se fue tarde

\begin{Shaded}
\begin{Highlighting}[]
\StringTok{\textquotesingle{}5.2.4{-}1.5\textquotesingle{}}\OtherTok{\textless{}{-}}\FunctionTok{filter}\NormalTok{ (nycflights13\_vu, arr\_delay }\SpecialCharTok{\textgreater{}} \DecValTok{2} \SpecialCharTok{\&}\NormalTok{ dep\_delay }\SpecialCharTok{\textless{}=} \DecValTok{0}\NormalTok{)}
\end{Highlighting}
\end{Shaded}

Se retrasaron al menos una hora, pero recuperaron más de 30 minutos en
vuelo

\begin{Shaded}
\begin{Highlighting}[]
\StringTok{\textquotesingle{}5.2.4{-}1.6\textquotesingle{}}\OtherTok{\textless{}{-}}\FunctionTok{filter}\NormalTok{ (nycflights13\_vu, dep\_delay }\SpecialCharTok{\textgreater{}=} \DecValTok{1} \SpecialCharTok{\&}\NormalTok{ arr\_delay }\SpecialCharTok{\textless{}=} \FloatTok{0.5}\NormalTok{)}
\end{Highlighting}
\end{Shaded}

Salida entre la medianoche y las 6 a. m. (incluidas)

\begin{Shaded}
\begin{Highlighting}[]
\StringTok{\textquotesingle{}5.2.4{-}1.7\textquotesingle{}}\OtherTok{\textless{}{-}}\FunctionTok{filter}\NormalTok{ (nycflights13\_vu, dep\_time }\SpecialCharTok{\textless{}=} \DecValTok{600}\NormalTok{)}
\end{Highlighting}
\end{Shaded}

para esta primera parte realizamos el uso de filters con rangos asinados
a traves de igualacion o mayor que, pero que sucederia si
implementaramos los mismo codigos con un filtro entre rangos:

\begin{Shaded}
\begin{Highlighting}[]
\CommentTok{\#con operador between}
\FunctionTok{filter}\NormalTok{ (nycflights13\_vu,}\FunctionTok{between}\NormalTok{( arr\_delay , }\StringTok{\textquotesingle{}2\textquotesingle{}}\NormalTok{ ,}\StringTok{\textquotesingle{}200\textquotesingle{}}\NormalTok{))}\DocumentationTok{\#\# para vuelos retrasados entre 2 y 200 hrs}
\end{Highlighting}
\end{Shaded}

\begin{verbatim}
## # A tibble: 125,129 x 19
##     year month   day dep_time sched_de~1 dep_d~2 arr_t~3 sched~4 arr_d~5 carrier
##    <int> <int> <int>    <int>      <int>   <dbl>   <int>   <int>   <dbl> <chr>  
##  1  2013     1     1      517        515       2     830     819      11 UA     
##  2  2013     1     1      533        529       4     850     830      20 UA     
##  3  2013     1     1      542        540       2     923     850      33 AA     
##  4  2013     1     1      554        558      -4     740     728      12 UA     
##  5  2013     1     1      555        600      -5     913     854      19 B6     
##  6  2013     1     1      558        600      -2     753     745       8 AA     
##  7  2013     1     1      558        600      -2     924     917       7 UA     
##  8  2013     1     1      559        600      -1     941     910      31 AA     
##  9  2013     1     1      600        600       0     837     825      12 MQ     
## 10  2013     1     1      602        605      -3     821     805      16 MQ     
## # ... with 125,119 more rows, 9 more variables: flight <int>, tailnum <chr>,
## #   origin <chr>, dest <chr>, air_time <dbl>, distance <dbl>, hour <dbl>,
## #   minute <dbl>, time_hour <dttm>, and abbreviated variable names
## #   1: sched_dep_time, 2: dep_delay, 3: arr_time, 4: sched_arr_time,
## #   5: arr_delay
\end{verbatim}

\begin{Shaded}
\begin{Highlighting}[]
\FunctionTok{filter}\NormalTok{ (nycflights13\_vu,}\FunctionTok{between}\NormalTok{(dest , }\StringTok{"IAH"}\NormalTok{ , }\StringTok{"HOU"}\NormalTok{ ))}\CommentTok{\# en este caso por el tipo de dato arrojara un error de coercion}
\end{Highlighting}
\end{Shaded}

\begin{verbatim}
## Warning in between(dest, "IAH", "HOU"): NAs introducidos por coerción

## Warning in between(dest, "IAH", "HOU"): NAs introducidos por coerción

## Warning in between(dest, "IAH", "HOU"): NAs introducidos por coerción
\end{verbatim}

\begin{verbatim}
## # A tibble: 0 x 19
## # ... with 19 variables: year <int>, month <int>, day <int>, dep_time <int>,
## #   sched_dep_time <int>, dep_delay <dbl>, arr_time <int>,
## #   sched_arr_time <int>, arr_delay <dbl>, carrier <chr>, flight <int>,
## #   tailnum <chr>, origin <chr>, dest <chr>, air_time <dbl>, distance <dbl>,
## #   hour <dbl>, minute <dbl>, time_hour <dttm>
\end{verbatim}

\begin{Shaded}
\begin{Highlighting}[]
\CommentTok{\#filter (nycflights13\_vu,between(carrier , "AA","UA","DL"))\# al trabajar con cadenas de caracteres se presenta el inconveniente de normalización}
\FunctionTok{filter}\NormalTok{ (nycflights13\_vu,}\FunctionTok{between}\NormalTok{(month,}\DecValTok{7}\NormalTok{,}\DecValTok{9}\NormalTok{))}\CommentTok{\#para el tema de meses años o dias, es perfecto el uso por que between representa un rango}
\end{Highlighting}
\end{Shaded}

\begin{verbatim}
## # A tibble: 86,326 x 19
##     year month   day dep_time sched_de~1 dep_d~2 arr_t~3 sched~4 arr_d~5 carrier
##    <int> <int> <int>    <int>      <int>   <dbl>   <int>   <int>   <dbl> <chr>  
##  1  2013     7     1        1       2029     212     236    2359     157 B6     
##  2  2013     7     1        2       2359       3     344     344       0 B6     
##  3  2013     7     1       29       2245     104     151       1     110 B6     
##  4  2013     7     1       43       2130     193     322      14     188 B6     
##  5  2013     7     1       44       2150     174     300     100     120 AA     
##  6  2013     7     1       46       2051     235     304    2358     186 B6     
##  7  2013     7     1       48       2001     287     308    2305     243 VX     
##  8  2013     7     1       58       2155     183     335      43     172 B6     
##  9  2013     7     1      100       2146     194     327      30     177 B6     
## 10  2013     7     1      100       2245     135     337     135     122 B6     
## # ... with 86,316 more rows, 9 more variables: flight <int>, tailnum <chr>,
## #   origin <chr>, dest <chr>, air_time <dbl>, distance <dbl>, hour <dbl>,
## #   minute <dbl>, time_hour <dttm>, and abbreviated variable names
## #   1: sched_dep_time, 2: dep_delay, 3: arr_time, 4: sched_arr_time,
## #   5: arr_delay
\end{verbatim}

\begin{Shaded}
\begin{Highlighting}[]
\FunctionTok{filter}\NormalTok{ (nycflights13\_vu,}\FunctionTok{between}\NormalTok{(arr\_delay, }\StringTok{\textquotesingle{}2\textquotesingle{}}\NormalTok{,}\StringTok{\textquotesingle{}200\textquotesingle{}}\NormalTok{) }\SpecialCharTok{\&}\NormalTok{ dep\_delay}\SpecialCharTok{\textless{}=}\StringTok{\textquotesingle{}0\textquotesingle{}}\NormalTok{)}\CommentTok{\#podemos usarlo como combinacion entre un valor exacto o con parametros y una comparacion con rangos}
\end{Highlighting}
\end{Shaded}

\begin{verbatim}
## # A tibble: 37,527 x 19
##     year month   day dep_time sched_de~1 dep_d~2 arr_t~3 sched~4 arr_d~5 carrier
##    <int> <int> <int>    <int>      <int>   <dbl>   <int>   <int>   <dbl> <chr>  
##  1  2013     1     1      554        558      -4     740     728      12 UA     
##  2  2013     1     1      555        600      -5     913     854      19 B6     
##  3  2013     1     1      558        600      -2     753     745       8 AA     
##  4  2013     1     1      558        600      -2     924     917       7 UA     
##  5  2013     1     1      559        600      -1     941     910      31 AA     
##  6  2013     1     1      600        600       0     837     825      12 MQ     
##  7  2013     1     1      602        605      -3     821     805      16 MQ     
##  8  2013     1     1      622        630      -8    1017    1014       3 US     
##  9  2013     1     1      624        630      -6     909     840      29 EV     
## 10  2013     1     1      624        630      -6     840     830      10 MQ     
## # ... with 37,517 more rows, 9 more variables: flight <int>, tailnum <chr>,
## #   origin <chr>, dest <chr>, air_time <dbl>, distance <dbl>, hour <dbl>,
## #   minute <dbl>, time_hour <dttm>, and abbreviated variable names
## #   1: sched_dep_time, 2: dep_delay, 3: arr_time, 4: sched_arr_time,
## #   5: arr_delay
\end{verbatim}

\begin{Shaded}
\begin{Highlighting}[]
\FunctionTok{filter}\NormalTok{ (nycflights13\_vu, }\FunctionTok{between}\NormalTok{( dep\_delay, }\StringTok{\textquotesingle{}1\textquotesingle{}}\NormalTok{,}\StringTok{\textquotesingle{}100\textquotesingle{}}\NormalTok{) }\SpecialCharTok{\&} \FunctionTok{between}\NormalTok{ (arr\_delay,}\DecValTok{0}\NormalTok{, }\FloatTok{0.5}\NormalTok{)) }\CommentTok{\# mismas combinaciones para rangos de enteros}
\end{Highlighting}
\end{Shaded}

\begin{verbatim}
## # A tibble: 1,921 x 19
##     year month   day dep_time sched_de~1 dep_d~2 arr_t~3 sched~4 arr_d~5 carrier
##    <int> <int> <int>    <int>      <int>   <dbl>   <int>   <int>   <dbl> <chr>  
##  1  2013     1     1     1240       1235       5    1415    1415       0 MQ     
##  2  2013     1     1     1832       1828       4    2144    2144       0 UA     
##  3  2013     1     1     1856       1855       1    2142    2142       0 DL     
##  4  2013     1     2      730        715      15    1206    1206       0 B6     
##  5  2013     1     2      954        945       9    1115    1115       0 WN     
##  6  2013     1     2     1023       1020       3    1330    1330       0 AA     
##  7  2013     1     2     1031       1015      16    1135    1135       0 UA     
##  8  2013     1     2     1526       1518       8    1823    1823       0 B6     
##  9  2013     1     2     1920       1910      10    2055    2055       0 MQ     
## 10  2013     1     2     1937       1900      37    2301    2301       0 DL     
## # ... with 1,911 more rows, 9 more variables: flight <int>, tailnum <chr>,
## #   origin <chr>, dest <chr>, air_time <dbl>, distance <dbl>, hour <dbl>,
## #   minute <dbl>, time_hour <dttm>, and abbreviated variable names
## #   1: sched_dep_time, 2: dep_delay, 3: arr_time, 4: sched_arr_time,
## #   5: arr_delay
\end{verbatim}

\begin{Shaded}
\begin{Highlighting}[]
\FunctionTok{filter}\NormalTok{ (nycflights13\_vu,}\FunctionTok{between}\NormalTok{( dep\_time , }\StringTok{\textquotesingle{}0\textquotesingle{}}\NormalTok{ , }\StringTok{\textquotesingle{}600\textquotesingle{}}\NormalTok{))}\CommentTok{\#podemos asumir que las 12 de la media noche son 00:00 y al ser enteros lo manejamos como 0}
\end{Highlighting}
\end{Shaded}

\begin{verbatim}
## # A tibble: 9,344 x 19
##     year month   day dep_time sched_de~1 dep_d~2 arr_t~3 sched~4 arr_d~5 carrier
##    <int> <int> <int>    <int>      <int>   <dbl>   <int>   <int>   <dbl> <chr>  
##  1  2013     1     1      517        515       2     830     819      11 UA     
##  2  2013     1     1      533        529       4     850     830      20 UA     
##  3  2013     1     1      542        540       2     923     850      33 AA     
##  4  2013     1     1      544        545      -1    1004    1022     -18 B6     
##  5  2013     1     1      554        600      -6     812     837     -25 DL     
##  6  2013     1     1      554        558      -4     740     728      12 UA     
##  7  2013     1     1      555        600      -5     913     854      19 B6     
##  8  2013     1     1      557        600      -3     709     723     -14 EV     
##  9  2013     1     1      557        600      -3     838     846      -8 B6     
## 10  2013     1     1      558        600      -2     753     745       8 AA     
## # ... with 9,334 more rows, 9 more variables: flight <int>, tailnum <chr>,
## #   origin <chr>, dest <chr>, air_time <dbl>, distance <dbl>, hour <dbl>,
## #   minute <dbl>, time_hour <dttm>, and abbreviated variable names
## #   1: sched_dep_time, 2: dep_delay, 3: arr_time, 4: sched_arr_time,
## #   5: arr_delay
\end{verbatim}

Realizamos las mismas consultas con la funcion filter, pero anidamos la
funcion between, nos percatamos que unicamente funcionan los rangos para
los numeros, en cuanto a caracteres o letras between no es el filtro
correcto para dichas sentencias.

\hypertarget{arrange}{%
\subsection{Arrange}\label{arrange}}

\#5.3.1

llamamos nuevamente la lista donde se encuentran los dataframes y
tydiverse para realizar nuestras consultas.

\begin{Shaded}
\begin{Highlighting}[]
\FunctionTok{library}\NormalTok{(nycflights13)}
\FunctionTok{library}\NormalTok{(tidyverse)}
\end{Highlighting}
\end{Shaded}

Debemos entender que informacion tiene el dataframe, para asi mismo
saber por que columnas y filas organizar, por ende implementamos la
funcion de colnames, para saber el orden del dataframe.

\begin{Shaded}
\begin{Highlighting}[]
\FunctionTok{colnames}\NormalTok{(nycflights13\_vu)}
\end{Highlighting}
\end{Shaded}

\begin{verbatim}
##  [1] "year"           "month"          "day"            "dep_time"      
##  [5] "sched_dep_time" "dep_delay"      "arr_time"       "sched_arr_time"
##  [9] "arr_delay"      "carrier"        "flight"         "tailnum"       
## [13] "origin"         "dest"           "air_time"       "distance"      
## [17] "hour"           "minute"         "time_hour"
\end{verbatim}

Esta sentencia nos entrega los encabezados de el dataframe

5.3.1 Ejercicios ¿Cómo podría utilizar arrange()para ordenar todos los
valores que faltan al principio? (Sugerencia: use is.na()).

\begin{Shaded}
\begin{Highlighting}[]
\StringTok{\textquotesingle{}5.3.1.1\textquotesingle{}}\OtherTok{\textless{}{-}}\NormalTok{ nycflights13\_vu}\SpecialCharTok{\%\textgreater{}\%}
\FunctionTok{arrange}\NormalTok{(}\FunctionTok{desc}\NormalTok{(}\FunctionTok{is.na}\NormalTok{(dep\_time))) }\CommentTok{\# en esta ocasion se organiza por tiempos de salida en NA}
\end{Highlighting}
\end{Shaded}

Ordenar flightspara encontrar los vuelos más retrasados. Encuentra los
vuelos que salieron antes.

\begin{Shaded}
\begin{Highlighting}[]
\StringTok{\textquotesingle{}5.3.1.2\textquotesingle{}}\OtherTok{\textless{}{-}}\NormalTok{ nycflights13\_vu}\SpecialCharTok{\%\textgreater{}\%}
\FunctionTok{arrange}\NormalTok{ (dep\_delay)}\CommentTok{\# }
\end{Highlighting}
\end{Shaded}

implementamos el la funcion para ver que vuelos salieron mas horas de
antelacion, y al final encontraremos los retrasados, a partir del
retraso en las llegadas

Ordenar flightspara encontrar los vuelos más rápidos (velocidad más
alta).

\begin{Shaded}
\begin{Highlighting}[]
\StringTok{\textquotesingle{}5.3.1.3\textquotesingle{}}\OtherTok{\textless{}{-}}\NormalTok{ nycflights13\_vu}\SpecialCharTok{\%\textgreater{}\%} \CommentTok{\#nos organiza en la parte superior los mas veloces, y al final los menos}
  \FunctionTok{arrange}\NormalTok{ (air\_time}\SpecialCharTok{*}\NormalTok{distance)}
\end{Highlighting}
\end{Shaded}

¿Qué vuelos viajaron más lejos? ¿Cuál viajó menos?

\begin{Shaded}
\begin{Highlighting}[]
\StringTok{\textquotesingle{}5.3.1.4\textquotesingle{}}\OtherTok{\textless{}{-}}\NormalTok{ nycflights13\_vu}\SpecialCharTok{\%\textgreater{}\%}\CommentTok{\#¿Qué vuelos viajaron más lejos? ¿Cuál viajó menos?}
  \FunctionTok{arrange}\NormalTok{ ((distance}\SpecialCharTok{/}\NormalTok{air\_time)}\SpecialCharTok{*}\DecValTok{60}\NormalTok{)}
\StringTok{\textquotesingle{}5.3.1.4{-}2 menos\textquotesingle{}} \OtherTok{\textless{}{-}}\NormalTok{ nycflights13\_vu}\SpecialCharTok{\%\textgreater{}\%}
  \FunctionTok{group\_by}\NormalTok{(distance)}\SpecialCharTok{\%\textgreater{}\%}
\FunctionTok{transmute}\NormalTok{(carrier,flight,tailnum,}\AttributeTok{velocidad\_promedio =}\NormalTok{ (distance}\SpecialCharTok{/}\NormalTok{air\_time)}\SpecialCharTok{*}\DecValTok{60}\NormalTok{)}\SpecialCharTok{\%\textgreater{}\%}
  \FunctionTok{ungroup}\NormalTok{()}
\end{Highlighting}
\end{Shaded}

Aqui implementamos una funcion especial, para poder determinar la
velocidad promedio y no solo asumir que por distancia y tiempo podiamos
guiarnos, por ende implementamos transmute.

\hypertarget{funcion-select}{%
\subsection{FUNCION SELECT}\label{funcion-select}}

¿Qué sucede si incluye el nombre de una variable varias veces en una
select()llamada?\#5.4.1.3 \#¿Qué hace la any\_of()función? ¿Por qué
podría ser útil junto con este vector?

\begin{Shaded}
\begin{Highlighting}[]
\CommentTok{\#vars \textless{}{-} c("year", "month", "day", "dep\_delay", "arr\_delay")\%\textgreater{}\%}
\CommentTok{\#SELECT ("dep\_delay",any\_of(dep\_delay=\textquotesingle{}na\textquotesingle{}))}
\end{Highlighting}
\end{Shaded}

lo que hace es coincidencias, y le dice a motor que coincida una o
varias veces con cualquiera\_de y se especifica el valor que se requiere
o el caracter entre comillas.

\#5.4.1.4\#¿Te sorprende el resultado de ejecutar el siguiente código?
¿Cómo tratan los ayudantes selectos el caso de forma predeterminada?
¿Cómo se puede cambiar ese valor predeterminado?

\begin{Shaded}
\begin{Highlighting}[]
\FunctionTok{select}\NormalTok{(flights, }\FunctionTok{contains}\NormalTok{(}\StringTok{"TIME"}\NormalTok{))}\CommentTok{\# en realidad lo que hace es generar un filtro en las columnas que tienen la palabra time como palabra clave}
\end{Highlighting}
\end{Shaded}

\begin{verbatim}
## # A tibble: 336,776 x 6
##    dep_time sched_dep_time arr_time sched_arr_time air_time time_hour          
##       <int>          <int>    <int>          <int>    <dbl> <dttm>             
##  1      517            515      830            819      227 2013-01-01 05:00:00
##  2      533            529      850            830      227 2013-01-01 05:00:00
##  3      542            540      923            850      160 2013-01-01 05:00:00
##  4      544            545     1004           1022      183 2013-01-01 05:00:00
##  5      554            600      812            837      116 2013-01-01 06:00:00
##  6      554            558      740            728      150 2013-01-01 05:00:00
##  7      555            600      913            854      158 2013-01-01 06:00:00
##  8      557            600      709            723       53 2013-01-01 06:00:00
##  9      557            600      838            846      140 2013-01-01 06:00:00
## 10      558            600      753            745      138 2013-01-01 06:00:00
## # ... with 336,766 more rows
\end{verbatim}

\begin{Shaded}
\begin{Highlighting}[]
\FunctionTok{select}\NormalTok{(flights, }\FunctionTok{contains}\NormalTok{(}\StringTok{"dep"}\NormalTok{))}\CommentTok{\#en este caso se usa el filtro para buscar la palabra dep en los encabezados de columnas}
\end{Highlighting}
\end{Shaded}

\begin{verbatim}
## # A tibble: 336,776 x 3
##    dep_time sched_dep_time dep_delay
##       <int>          <int>     <dbl>
##  1      517            515         2
##  2      533            529         4
##  3      542            540         2
##  4      544            545        -1
##  5      554            600        -6
##  6      554            558        -4
##  7      555            600        -5
##  8      557            600        -3
##  9      557            600        -3
## 10      558            600        -2
## # ... with 336,766 more rows
\end{verbatim}

en realidad solo trae una unica fila a pesar de ser llamada 3 veces

\begin{Shaded}
\begin{Highlighting}[]
\CommentTok{\#\textquotesingle{}5.4.1.2\textquotesingle{}\textless{}{-} nycflights13\_vu\%\textgreater{}\%}

 \CommentTok{\#select(carrier,carrier,carrier,flight,tailnum)\%\textgreater{}\%}

\CommentTok{\#arrange(carrier)}
\end{Highlighting}
\end{Shaded}

Aqui implementamos funciones adiciona

\hypertarget{mutate}{%
\subsection{MUTATE}\label{mutate}}

Ejercicios 5.5.2

Actualmente dep\_time y sched\_dep\_time son convenientes a la vista,
pero difíciles de calcular porque en realidad no son números continuos.
Conviértalos a una representación más conveniente de la cantidad de
minutos desde la medianoche.

\begin{Shaded}
\begin{Highlighting}[]
\StringTok{\textquotesingle{}5.5.2.1\textquotesingle{}} \OtherTok{\textless{}{-}}\NormalTok{ nycflights13\_vu}\SpecialCharTok{\%\textgreater{}\%}
\FunctionTok{transmute}\NormalTok{(dep\_time,sched\_dep\_time,}\AttributeTok{dep\_horas =}\NormalTok{(dep\_time }\SpecialCharTok{\%/\%} \DecValTok{100}\NormalTok{), }\AttributeTok{dep\_minutos =}\NormalTok{ (dep\_time }\SpecialCharTok{\%\%} \DecValTok{100}\NormalTok{ ),}\AttributeTok{sched\_horas=}\NormalTok{(sched\_dep\_time}\SpecialCharTok{\%/\%} \DecValTok{100}\NormalTok{),}\AttributeTok{sched\_minutos =}\NormalTok{ (sched\_dep\_time }\SpecialCharTok{\%\%} \DecValTok{100}\NormalTok{ ))}
\end{Highlighting}
\end{Shaded}

Comparar air\_timecon arr\_time - dep\_time. Que esperas ver? ¿Que ves?
¿Qué necesitas hacer para arreglarlo?

\begin{Shaded}
\begin{Highlighting}[]
\StringTok{\textquotesingle{}5.5.2.2\textquotesingle{}} \OtherTok{\textless{}{-}}\NormalTok{ nycflights13\_vu}\SpecialCharTok{\%\textgreater{}\%}
\FunctionTok{select}\NormalTok{(air\_time,arr\_time,dep\_time)}\SpecialCharTok{\%\textgreater{}\%}
\FunctionTok{group\_by}\NormalTok{(arr\_time)}
\end{Highlighting}
\end{Shaded}

en primera medida se ve que el air time esta en minutos, debo
convertirlo a horas.

procuro dejar todo en horas para que sea mas facil la lectura y el
calculo correspondiente

\begin{Shaded}
\begin{Highlighting}[]
\CommentTok{\#transmute(hora\_llegada=(arr\_time/100),hora\_salida=(dep\_time/100),tiempo\_vuelo\_horas=(air\_time/60),arr\_time)\%\textgreater{}\%}
\CommentTok{\#ungroup\%\textgreater{}\%}
\CommentTok{\#arrange(arr\_time)}
\end{Highlighting}
\end{Shaded}

\hypertarget{sumarise}{%
\subsection{SUMARISE}\label{sumarise}}

Ejercicios 5.6.7

Haga una lluvia de ideas sobre al menos 5 formas diferentes de evaluar
las características típicas de retraso de un grupo de vuelos. Considere
los siguientes escenarios:

\#Un vuelo llega 15 minutos antes el 50\% del tiempo y 15 minutos tarde
el 50\% del tiempo

\begin{Shaded}
\begin{Highlighting}[]
\NormalTok{not\_cancelled \%\textgreater{}\%}
\NormalTok{  group\_by(year, month, day) \%\textgreater{}\%}
\NormalTok{  summarise(n\_early = sum(dep\_time)+Tiempo\_adicional)}
\end{Highlighting}
\end{Shaded}

teniendo en cuenta que la hora de llegada de los vuelos sera aumentada y
disminuida en todos los casos implementamos una resta y suma segun
corresponda de 15 minutos.

\begin{Shaded}
\begin{Highlighting}[]
\CommentTok{\#Tiempo\_adicional\textless{}{-}15}
\CommentTok{\#not\_cancelled \%\textgreater{}\%}
\CommentTok{\#  group\_by(year, month, day) \%\textgreater{}\%}
\CommentTok{\#  summarise(n\_early = sum(dep\_time){-}Tiempo\_adicional)}
\end{Highlighting}
\end{Shaded}

Un vuelo siempre llega 10 minutos tarde.

\begin{Shaded}
\begin{Highlighting}[]
\CommentTok{\#\textquotesingle{}10\_TARDE\textquotesingle{}\textless{}{-}not\_cancelled \%\textgreater{}\%}
\CommentTok{\#  group\_by(year, month, day)\%\textgreater{}\%}
\CommentTok{\#transmute(flight,tailnum,origin,dest,sched\_arr\_time,hora\_con\_retraso=(sched\_arr\_time + 10))\%\textgreater{}\%}
\CommentTok{\#ungroup}
\end{Highlighting}
\end{Shaded}

Siempre habra un retraso de 10 minutos entonces incrementamos el tiempo
de llegada estimado, con eso ya sabemos que estara retrasado con
respecto a la hora dispuesta.

Un vuelo llega 30 minutos antes el 50\% del tiempo y 30 minutos tarde el
50\% del tiempo.

\begin{Shaded}
\begin{Highlighting}[]
\CommentTok{\#\textquotesingle{}30\_mminutos\textquotesingle{}\textless{}{-}not\_cancelled \%\textgreater{}\%}
\CommentTok{\#  group\_by(year, month, day)\%\textgreater{}\%}
\CommentTok{\#  transmute(flight,tailnum,origin,dest,sched\_arr\_time,hora\_adelanto=sched\_arr\_time{-}30,hora\_atraso=sched\_arr\_time+30)\%\textgreater{}\%}
\CommentTok{\#  transmute(Tiempo\_desviacion=(mean(hora\_atraso{-}hora\_adelanto)))\%\textgreater{}\%}
\CommentTok{\#  ungroup}
\end{Highlighting}
\end{Shaded}

implementamos una funcion donde se contemplen las dos llegadas, una con
retraso, y la otra con adelanto con respecto al tiempo estimado,
asumiendo que es el tiempo idoneo de llegada.

\hypertarget{group_by-agrupamientos}{%
\subsection{GROUP\_BY AGRUPAMIENTOS}\label{group_by-agrupamientos}}

ejercicios 5.7.1

Vuelva a consultar las listas de funciones útiles de mutación y
filtrado. Describe cómo cambia cada operación cuando la combinas con la
agrupación.

\begin{Shaded}
\begin{Highlighting}[]
\CommentTok{\#5.7.1 Vuelva a consultar las listas de funciones útiles de mutación y filtrado. Describe cómo cambia cada operación cuando la combinas con la agrupación.}
\NormalTok{avion\_retraso }\OtherTok{\textless{}{-}}\NormalTok{ nycflights13\_vu}\SpecialCharTok{\%\textgreater{}\%}
\FunctionTok{arrange}\NormalTok{ (flights,}\FunctionTok{desc}\NormalTok{(dep\_delay))}\SpecialCharTok{\%\textgreater{}\%}
\FunctionTok{group\_by}\NormalTok{(tailnum)}
\end{Highlighting}
\end{Shaded}

Lo que hacemos es organizar el dep\_delay(retraso en salidas) que
equivale a el retraso con respecto a la salida estimada, y la hora de
salida y luego agrupamos por el numero en la cola del avion, aqui lo que
realiza es un ordenamiento a traves de los numeros de la cola del avion
permitiendonos tener un consolidado de informacion sobre los mismos.

La documentación tanto para el taller como el material academico del
cual fuimos apoyados fue tomado de:

\href{https://r4ds.had.co.nz/transform.html}{Documentacion para R
(inglés)}

\href{https://github.com/ErikCastillor/entregar1ercorte}{Repositorio
Erik}

\end{document}
